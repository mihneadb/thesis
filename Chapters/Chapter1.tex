% Chapter 1

\chapter{Introduction} % Main chapter title

\label{Chapter1} % For referencing the chapter elsewhere, use \ref{Chapter1} 

\lhead{Chapter 1. \emph{Introduction}} % This is for the header on each page - perhaps a shortened title

%----------------------------------------------------------------------------------------

Many-Task Computing (MTC)\cite{mtc} is a new computing paradigm that attempts to place itself in-between High-Throughput Computing (HTC) and High-Performance Computing (HPC). The core idea of MTC is that it uses large amounts of computing resources over short periods of time, in order to complete multiple tasks. While a common metric in HTC is the number of tasks fulfilled per month, MTC is usually quantified in tasks per second. Also, as opposed to other paradigms that use message passing as a way to communicate between successive steps in a computation, MTC couples its compute phases via files. Because of this, another common metric of MTC is megabytes per second (MB/s).

Any computing task that is comprised of multiple small parallel jobs is a good fit for MTC. Multiple applications have been found good fits for MTC\cite{mtc}, spanning domains like astronomy, astrophysics, economic modelling, pharmaceutics, chemistry, bioinformatics, neuroscience, data analytics, data mining and biometrics.

MTC exhibits a specific set of technical particularities which need to be properly supported in order to obtain good performance. The high number of parallel tasks that need to be executed underlines the importance of a streamlined scheduling system. Besides that, the way the tasks use files to communicate makes the overall performance largely dependant on the underlying filesystem performance. For the purpose of our subject, we will focus on the filesystem aspect.



** how MTC filesystems work

** memfs
** amfs

** why benchmarking them is different (access patterns) - explicit