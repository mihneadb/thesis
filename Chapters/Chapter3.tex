% Chapter Template

\chapter{A Benchmark for MTC Filesystems} % Main chapter title

\label{Chapter3} % Change X to a consecutive number; for referencing this chapter elsewhere, use \ref{ChapterX}

\lhead{Chapter 3. \emph{A Benchmark for MTC Filesystems}} % Change X to a consecutive number; this is for the header on each page - perhaps a shortened title

In the previous two chapters we have discussed about Many-Task Computing, how it is relevant for today's  computations and why it needs special filesystems. We looked at how those filesystems differ from regular disk-based ones, as well as why they need to be benchmarked differently. We also outlined the shortcomings of current benchmarks and how they can be surpassed. In this chapter, we focus on our approach to building an MTC-specific filesystem benchmark, the architecture behind it and the ways in which it can be used.


* leverage iozone and mdtest
* parsers for the two
* plotting

* how to coordinate any number of nodes for any command

* stdout problem

* how to automatically run with varying number of nodes

* automate the cluster reservation process

* specifying complex tests

** architecture


** support standard MPI
** cluster agnostic
