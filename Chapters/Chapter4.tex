% Chapter Template

\chapter{Evaluation} % Main chapter title

\label{Chapter4} % Change X to a consecutive number; for referencing this chapter elsewhere, use \ref{ChapterX}

\lhead{Chapter 4. \emph{Evaluation}} % Change X to a consecutive number; this is for the header on each page - perhaps a shortened title

To evaluate our solution, we follow two directions. First, we test the fact that our benchmark supports MTC filesystems and that it can measure the metrics described by the MTC Envelope. Second, we observe whether our project is easier to use than the current benchmarking solutions for MTC filesystems.

To assess the first part, we ran the following benchmarks on the two discussed MTC filesystems - MemFS and AMFS.

\begin{itemize}

\item 1-to-1 read
\item N-to-1 read
\item mdtest suite

\end{itemize}

All the tests were run on the following numbers of nodes: \texttt{[1, 2, 4, 8, 16, 32]}. The benchmarks were run on the DAS-4 cluster, using OpenMPI.

The N-to-1 access pattern was simulated using a test file such as the following:\\\\

\lstinputlisting[caption=Sample test file]{Files/test_example2.json}

Having this file stored as \textit{test.json}, we can run the benchmark with:

\begin{verbatim}
./dfs_bench.py --file test.json
\end{verbatim}

The full set of results can be found in the Appendix. However, we include here a sample results file and a plot to show the outputs of our benchmark.

The following listing represents a part of a results file of an IOzone-based benchmark, testing N-to-1 read. It depicts the processed result set for the test run with 2 nodes.

\begin{verbatim}
{
  "2": {
    "total": {
      "re-reader": 1433371, 
      "reader": 1345494
    }, 
    "individual": [
      {
        "re-reader": 728699, 
        "reader": 721448
      }, 
      {
        "re-reader": 704672, 
        "reader": 624046
      }
    ]
  }
}
\end{verbatim}

Based on a processed set of results, the \textit{make\_bar\_plots.py} script creates a plot like the following.

\begin{figure}[H]
  \centering
    \includegraphics[scale=0.5]{Figures/reader.png}
    \rule{25em}{0.5pt}
  \caption[Generated sample plot]{Generated sample plot}
  \label{fig:architecture}
\end{figure}

we can see .. just one command , scaling taken care of, coordination, aggregation, plots
